\documentclass[a4paper, 10pt]{article}
\usepackage[utf8]{inputenc}
\usepackage[spanish]{babel}
\usepackage{graphicx}
\usepackage{geometry}
\usepackage{listings}
\usepackage{amsmath}
\usepackage{amsfonts}
\usepackage{amssymb}
\usepackage{caratula}

\newcommand{\Z}{\mathbb{Z}}
\def\code#1{\texttt{#1}}
\newcommand\tab[1][0.5cm]{\hspace*{#1}}

\geometry{a4paper, margin=0.7in}

\begin{document}
    %Caratula
    \pagenumbering{gobble}
    \newpage

    \begin{center}
        \includegraphics[width=7.5cm, height=7.5cm]{images/logo}
    \end{center}

    \materia{Organización de Datos}
    \submateria{Segundo Cuatrimestre 2017}
    \titulo{Trabajo Práctico 1}

    \integrante{Rodrigo De Rosa}{97799}{rodrigoderosa@outlook.com}
    \integrante{Marcos Schapira}{97934}{schapiramarcos@gmail.com}
    \integrante{Facundo Guerrero}{97981}{facundoiguerrero@gmail.com}
    \maketitle
    %Fin caratula
    %Table of contents
    \newpage
    \pagenumbering{roman}
    \tableofcontents
    %Fin table of contents
    %Informe
    \newpage
	\pagenumbering{arabic}
	\part{Análisis del precio por $m^2$}
		\section{Adaptación del DataFrame}
			Para el análisis particular de cada característica de la información que se posee, se adaptó el DataFrame
			original para poder analizar dicha información mas fácil y comodamente.
			\subsection{Filtrado de columnas}
				Para el análisis de esta cierta característica de las propiedades, consideramos \emph{importantes}
				sólo a algunas celdas. Estas son:
				\begin{itemize}
					\item \code{place name} $\leftarrow$ \code{location}
					\item \code{price aprox usd} $\leftarrow$ \code{price}
					\item \code{surface total in m2} $\leftarrow$ \code{totalSurface}
					\item \code{surface covered in m2} $\leftarrow$ \code{coveredSurface}
					\item \code{price usd per m2} $\leftarrow$ \code{pricem2}		
				\end{itemize}
			\subsection{Completando el DataFrame}
				Lo primero que se hizo para realizar este analisis fue completar las columnas faltantes
				de la mayor cantidad de entradas posibles. Esto es, \code{location, price, totalSurface,
				pricem2}. De esta forma, nos permitimos analizar una mayor cantidad de propiedades
				para realizar un analisis un poco mas correcto. \\
				\tab Para completar el campo de precio por $m^2$ se necesita que la entrada sobre la que
				se trabaja cumpla la siguiente condición lógica: $price_{}m^2 \vee (price \wedge surface)$. Es
				decir, necesita tener o el precio por metro cuadrado o tanto el precio total como la superficie
				total. \\
				\tab Si el campo \code{pricem2} tiene valor, entonces ese será el utilizado. En caso contrario,
				si tanto el campo \code{price} como el campo \code{totalSurface} tienen valor, definimos como
				nuestro nuevo \code{pricem2} a la división $\frac{price}{surface}$. \\
				\tab Para esto, necesitamos unificar \code{coveredSurface} y \code{totalSurface}, para maximizar
				nuevamente la cantidad de entradas disponibles. Esto se hace, simplemente, poniendo como
				\code{totalSurface} el valor de \code{coveredSurface} en aquellas entradas donde la primera
				no tenga valor (consideramos que $total - covered = uncovered$). \\
				\tab Una vez completados todos los \code{pricem2} posibles, eliminamos todas aquellas entradas
				que tengan \code{NaN} como valor (en cualquiera de las celdas que definimos como \emph{importantes}),
				pues ya no podemos obtener el valor de esa celda de ningun otro lugar.
		\section{Estudio estadístico de los datos}
			\subsection{Analisis de la distribución de precios}
				Una vez completado el DataFrame lo mas posible, se realizó un análisis de la distribución de precios.
				Con esto nos referimos a analizar la variación del precio por metro cuadrado entre todas las
				propiedades. Es decir, \emph{limpiar los datos que no tienen sentido}. \\
				\tab Para esto le pedimos el \code{.describe()} a nuesto DataFrame con los percentiles $0.01$ y
				$0.99$. Esto nos permite analizar que tan desviados estan los valores máximos y mínimos. \\
				\tab Con los percentiles recién mencionados hacemos un recorte de los datos para lograr una distribución
				que se asemeje a una Normal lo mas posible. El primer recorte es tanto inferior ($ > 150$USD) como superior
				($<18000$USD). Como en este nuevo DataFrame la diferencia entre el percentil $0.99$ y el máximo es de más
				del doble, se vuelve a recortar superiormente ($<8000$USD). \\
				\tab Luego de esto, la distribución de precios es un poco mejor que antes (las diferencias entre los
				percentiles $0.25, 0.5, 0.75$ son similares). \\
				\tab A continuación se muestra un gráfico de distribución de precios por metro cuadrado que se obtiene
				del DataFrame original sin realizar el filtrado recién mencionado. Nos hubiese gustado poder mostrar tanto
				el KDE como el histograma pero al haber tanta diferencia entre el maximo y los valores principales de la
				distribución, el histograma era solo una linea. El objetivo de este gráfico es hacer incapié en lo mencionado
				en el previo párrafo: es necesario filtrar los datos para tener un conjunto de datos con sentido.
				\begin{center}
       				\includegraphics[width=6in, height=4.2in]{images/m2UnfilteredKDE}
		   		\end{center}
				\tab En el siguiente gráfico de distribución de precios por metro cuadrado se puede ver que la mayor parte
				de las propiedades están concentradas en el rango de precios $[150;4000]$USD y luego hay un drástico decaimiento
				de cantidad de propiedades para el resto de los precios. Si bien se podría considerar que un recorte sería
				correcto, a partir de fuentes externas se sabe que ciertos barrios (\emph{i.e.} Puerto Madero) tienen,
				aproximadamente, un valor medio de $6000$USD por metro cuadrado.
				\begin{center}
       				\includegraphics[width=6in, height=4.2in]{images/m2Histogram}
		   		\end{center}
			\subsection{Agrupando por barrios}
				Ahora que nuestros datos están tan completos y retocados como querríamos, procedemos a agrupar todas las propiedades
				de acuerdo al barrio al que pertencen. Una vez que los tenemos agrupados, debemos establecer un \emph{minimo de
				propiedades} por barrio. Pues un barrio que tiene una o dos propiedades podría alterar el estudio de la
				informacion. \\
				\tab Nuevamente, para esta tarea utilizizamos \code{.describe()} y resolvemos que utilizaremos como cota inferior
				$50$ propiedades (dos mas que el equivalente a una publicación por mes en los últimos cuatro años). \\
				\tab Aquí, al igual que hicimos antes, mostraremos la distribución antes y después del filtro aplicado. Si bien
				en escencia no son tan diferentes, podemos observar que desaparecen algunos barrios de la zona de precios altos.
				\begin{center}
       				\includegraphics[width=6in, height=4.2in]{images/m2HoodUnfilteredKDE}
		   		\end{center}
				\tab Una vez que eliminamos los barrios problemáticos, si analizamos la distribución de precio por barrio
				podemos ver que la mayor parte está concentrada en el intervalo $[500;3500]$USD, mientras que muy pocos
				(solo tres) superan ese valor.
				\begin{center}
   	    				\includegraphics[width=6in, height=4.2in]{images/m2HoodHistogram}
			    \end{center}	
			    \tab Podemos ver, además, que la distribución es bastante similar a la anterior (sin agrupar por barrios) aunque,
			    obviamente, con valores menores (pues son promedios).
		\section{Analizando grupos característicos}
			En esta sección analizaremos ciertos grupos característicos a partir de la información con la que estamos trabajando.
			\subsection{Los diez barrios con mayor precio por $m^2$}
				Dado que ya estamos felices con la forma en que tenemos dispuestos los datos, comenzaremos por hacer un
				\emph{Top 10} de los barrios más caros de CABA y GBA. \\
				\tab Para esto, como ya tenemos los datos agrupados, simplemente ordenamos el DataFrame y nos quedamos con
				los primeros diez.
				\tab Durante el análisis de esta información, notamos que varios de los barrios que aparecían en este \emph{Top 10}
				eran subdivisiones del barrio de Palermo. Por esta razón, decidimos incluir dos casos: uno en que consideramos
				que todos los 'Palermos' son uno solo, y otro en que cada uno es considerado un barrio diferente.
				\subsubsection{Unificación de Palermo}
					En este caso, consideramos que todas las subdivisiones de Palermo pertenecen a un sólo barrio.\\
					\tab El resultado obtenido es el siguiente:
					\begin{center}
						\begin{tabular}{ |c|c|c| }
							\hline
							\multicolumn{3}{|c|}{Top 10 [Palermo unificado]}\\
							\hline
							\hline
							Puesto & Barrio & Precio $m^2$ [U$\$$D] \\
							\hline
							1 & Puerto Madero & 5657 \\
							2 & Las Cañitas & 3612 \\
							3 & Palermo & 3518 \\
							4 & Recoleta & 3316 \\
							5 & Belgrano & 3124 \\
							6 & Nuñez & 3056 \\
							7 & Barrio Norte & 2949 \\
							8 & Vicente López & 2925 \\
							9 & Retiro & 2783 \\
							10 & Olivos & 2737 \\
							\hline
						\end{tabular}
					\end{center}
					\tab En la tabla se observa que Puerto Madero tiene un valor mucho mas alto que el resto, de hecho, es
					mayor al doble del precio del décimo. De todos modos, entre el segundo y el último la variación es más
					suave. Para aportar a este análisis, se realiza un gráfico de barras:
					\begin{center}
   	    					\includegraphics[width=6in, height=3.9in]{images/m2UnifiedTop10}
			  		\end{center}	
			  	\subsubsection{División de Palermo}
			  		Aquí consideraremos que el barrio al que se nombra Palermo corresponde a todas las secciones de dicho
			  		barrio que no son las que ya aparecen en otro grupo. \\
			  		\tab En este caso, el resultado obtenido es:
			  		\begin{center}
						\begin{tabular}{ |c|c|c| }
							\hline
							\multicolumn{3}{|c|}{Top 10 [Palermo dividido]}\\
							\hline
							\hline
							Puesto & Barrio & Precio $m^2$ [U$\$$D] \\
							\hline
							1 & Puerto Madero & 5657 \\
							2 & Palermo Chico & 4489 \\
							3 & Las Cañitas & 3612  \\
							4 & Palermo Viejo & 3419 \\
							5 & Recoleta & 3316 \\
							6 & Palermo & 3260 \\
							7 & Palermo Hollywood & 3224 \\
							8 & Palermo Soho & 3198 \\
							9 & Belgrano & 3124 \\
							10 & Nuñez & 3056 \\
							\hline
						\end{tabular}
					\end{center}
					\tab En la tabla podemos ver que, si bien es correcto y es un \emph{Top 10}, esta plagado de subdivisiones
					de Palermo y no nos permite tener un plano más general. \\
					\tab Aquí el gráfico de barras es muy similar aunque aparece Palermo Chico, que se acerca un poco mas
					al valor de Puerto Madero. De todos modos, la diferencia entre el primero y el segundo es muy grande
					como también lo es entre el segundo y el tercero, dejando la relación entre los valores igual de 'no suave'.
					\begin{center}
   	    					\includegraphics[width=6in, height=3.9in]{images/m2NotUnifiedTop10}
			  		\end{center}
			  		\tab De aquí en más, utilizaremos a Palermo como un barrio unificado.
			  	\subsubsection{Comentario sobre el Top 10}
				  	Este \emph{Top 10} arroja los resultados que se hubieran esperado, pues los únicos dos valores que no pertenecen
			  	a CABA corresponden a los primeros dos barrios de GBA en los que se piensa al pensar en los barrios mas caros
			  	de Buenos Aires. \\
			  		\tab Por otro lado, si nos sorprende el hecho de que el $m^2$ en Barrio Norte sea más barato que Núñez o en
			  		Belgrano.
			\subsection{Los diez barrios con menor precio por $m^2$}
				Para esta parte, al igual que antes, ordenamos los datos para analizar cuales son los diez barrios que se encuentran
				en el \emph{Bottom 10}. \\
				\tab Al realizar este análisis, lo obtenido es:
				\begin{center}
					\begin{tabular}{ |c|c|c| }
						\hline
						\multicolumn{3}{|c|}{Bottom 10}\\
						\hline
						\hline
						Puesto & Barrio & Precio $m^2$ [U$\$$D] \\
						\hline
						1 & Ingeniero Pablo Nogués & 494 \\
						2 & Maschwitz & 548 \\
						3 & Villa Udaondo & 558 \\
						4 & Parque Leloir & 579 \\
						5 & Presidente Perón & 643 \\
						6 & Benavidez & 705 \\
						7 & José C Paz & 711 \\
						8 & Villa Libertad & 743 \\
						9 & Longchamps & 800 \\
						10 & Burzaco & 804 \\
						\hline
					\end{tabular}
				\end{center}
				\tab Si graficamos estos valores al igual que antes podremos ver un ascenso (o descenso) más suave que el del
				\emph{Top 10}. Si bien el primero es casi la mitad de el último, la variación entre puestos es menor. \\
				\begin{center}
   	    				\includegraphics[width=6in, height=3.9in]{images/m2Bottom10}
			  	\end{center}
				\tab Aquí, remitiéndonos a la sección 3.1.3, vemos que los barrios del \emph{Bottom 10} son todos barrios
				alejados de la ciudad, de los cuales es esperable un bajo valor del $m^2$.
			\subsection{Dividiendo en secciones}
				El objetivo de esta parte es determinar diferentes grupos de barrios basados en el precio promedio del $m^2$
				para, de esta manera, analizar que tan suave (o no) es el decrecimiento del valor del $m^2$ en cada uno de
				estos grupos. \\
				\tab Los barrios serán divididos en grupos a partir de valores arbitrarios de precio por $m^2$, que surgen de
				un análisis de los datos. Los grupos serán:
				\begin{center}
					\begin{tabular}{ |c|c|c| }
						\hline
						\multicolumn{3}{|c|}{Divisones} \\
						\hline
						\hline
						Numero & min(price$m^2$) [USD] & max(price$m^2$) [USD] \\
						\hline
						1 & $2500$ & $\infty$ \\
						\hline
						2 & $2000$ & $2499$ \\
						\hline
						3 & $1500$ & $1999$ \\	
						\hline
						4 & $1200$ & $1499$ \\
						\hline
						5 & $950$ & $1199$ \\		
						\hline
						6 & $450$ & $949$ \\			
						\hline
					\end{tabular}
				\end{center}
				\tab El siguiente gráfico de área muestra el precio del $m^2$ y las divisiones (diferentes intensidades de
				azul) indican los diferentes grupos.
				\begin{center}
   	    				\includegraphics[width=6in, height=3.6in]{images/m2HoodsDivision}
			  	\end{center}
			  	\tab Nos interesa mostrar también que porcentaje de los barrios está incluído en cada uno de estos grupos para
			  	saber en cuál de ellos se concentra la mayor parte.
			  	\begin{center}
			  		\begin{tabular}{ |c|c|c| }
						\hline
						\multicolumn{3}{|c|}{Distribución en grupos} \\
						\hline
						\hline
						Grupo & Cantidad de barrios & Porcentaje \\
						\hline
						1 & 16 & $9\%$ \\
						\hline
						2 & 28 & $16\%$ \\
						\hline
						3 & 48 & $27\%$ \\
						\hline
						4 & 45 & $26\%$ \\
						\hline
						5 & 23 & $13\%$ \\
						\hline						
						6 & 16 & $9\%$ \\
						\hline						
			  		\end{tabular}
				\end{center}			  	 
				\tab Vemos que, como se puede apreciar en el gráfico, los grupos $3$ y $4$ son los que concentran a la mayor
				cantidad de barrios. De hecho, entre ellos dos tienen a más del $50\%$. \\
				\tab Ahora procedermos a analizar grupo por grupo.
				\subsubsection{Grupo 1 - $[2500:\infty)$U$\$$D}
					En este grupo se encuentra a los dieciséis barrios más caros de CABA y GBA. Aunque a los primeros diez ya los
					conocemos de la sección 3.1, lo que nos interesa en esta parte es analizar y comparar cómo varía el precio
					del metro cuadrado grupo a grupo mas que los nombres propios de cada integrante de cada grupo. En cada uno
					de los grupos se mostrará un \emph{zoom} del gráfico de área previo, para agregar una visualización al
					análisis de la variación en cada caso. \\
					\tab En este caso, lo que esperamos encontrar es un inicio con una pendiente muy inclinada y, a medida que
					nos alejamos del primer barrio (que ya sabemos que es Puerto Madero), un decrecimiento de dicha pendiente,
					pues la diferencia entre barrio y barrio será mucho menor, aunque seguirá siendo el grupo con mayor variación
					entre barrios de todos.
					\begin{center}
   		    				\includegraphics[width=6in, height=3.6in]{images/m2Group1Area}
				  	\end{center}
				  	\tab Si bien el gráfico muestra lo que se esperaba, también se ve que a partir del doceavo barrio la curva
				  	se vuelve casi constante (en verde), mostrando que hay un cierto subgrupo de barrios con precios muy 
				  	similares. Esto se puede ver en el gráfico de la sección anterior si se observa el pequeño valle que 
				  	se forma justo antes 	de llegar a la división entre el grupo 1 y el 2. \\
				  	\tab Cabe destacar, además, que si no fuera por el precio extraordinario de Puerto Madero, el grupo tendría
				  	una variación mucho menor, pues iría entre $3600$USD y $2500$USD lo que representaría una variación de
				  	$1100$USD ($30\%$) entre el máximo y el mínimo mientras que, actualmente, la variación es de $3100$USD
				  	($55\%$).
				\subsubsection{Grupo 2 - $[2000:2500)$U$\$$D}
					Observando el gráfico en el que se realizaron las divisiones por grupo esperamos que este segundo grupo
					tenga un sector medio con pendiente casi nula, donde la diferencia entre el precio en un barrio y el
					siguiente sea casi nula.
					\begin{center}
   		    				\includegraphics[width=6in, height=4.13in]{images/m2Group2Area}
				  	\end{center}
				  	\tab El sector indicado el violeta es aquel que mencionamos previamente con pendiente casi nula. En este
				  	subgrupo se encuentran diez barrios y la diferencia entre el primero y el último es de solamente $46$USD.
				  	Dado los numeros que se manejan, esa diferencia es casi despreciable, teniendo así un subgrupo de diez 
				  	barrios con el mismo valor para el $m^2$. \\
				  	\tab En cuanto al porcentaje de variación, a partir de este segundo grupo lo podemos dividir en dos; uno
				  	particular del grupo, en donde el porcentaje se calcula a partir del máximo local, y otro general, en donde
				  	el porcentaje se calcula a partir del máximo global (\emph{i.e.} Puerto Madero). La diferencia entre el
				  	máximo local y el mínimo es de $422$USD, siendo, de esta manera, la variación local del $17\%$ y la general
				  	del $7\%$. \\
				  	\tab Vemos entonces que la diferencia de variación ya decreció en gran medida pasando del grupo 1 al grupo 2.
				\subsubsection{Grupo 3 - $[1500:2000)$U$\$$D}
					Ahora llegamos al primero de los dos grupos principales. Como vimos antes, este grupo concentra al $27\%$ de
					los barrios de CABA y GBA pero (según nos muestra a grandes rasgos el gráfico general) con mayor variación
					que el próximo grupo, que también concentra a una gran parte de los barrios.\\
					\tab En este caso, esperamos ver una pendiente casi constante en todo el grupo sin regiones con pendiente
					casi nula pero con una variación pequeña entre un barrio y el siguiente.
					\begin{center}
   		    				\includegraphics[width=6in, height=4.13in]{images/m2Group3Area}
				  	\end{center}
				  	\tab En este gráfico se observa lo previamente indicado, el decrecimiento es bastante lineal y no se detectan
				  	zonas con precios constantes. De todas formas, si se amplía un poco, se pueden encontrar pares o tríos de grupos
				  	que tienen precios similares, como se ve a continuación.
				  	\begin{center}
   		    				\includegraphics[width=6in, height=4.13in]{images/m2Group3Detail}
				  	\end{center}
				  	\tab Aquí podemos ver que si bien hay diecisiete barrios que comparten precio con algún otro, no son contiguos
				  	como en el grupo 2. De todas maneras, como en este grupo la variación general de precio es menor y la cantidad
				  	de barrios es mayor, podríamos encontrar también un grupo de diez barrios con variación de aproximadamente
				  	$50$USD. Por ejemplo, entre Monte Castro y Ramos Mejía hay una diferencia de $60$USD y se encierra un subgrupo
				  	de nueve barrios. \\
				  	\tab Analizando el porcentaje de variación al igual que en el Grupo 2, sabiendo que la diferencia entre el
				  	máximo local y el mínimo es de $493$USD, el porcentaje de variación local es de un $25\%$ y el porcentaje
				  	de variación global es del $9\%$. \\
				  	\tab Se observa entonces que, lógicamente, ahora que estamos en una zona de precios menores, una diferencia
				  	entre mínimo y máximo similar a la del grupo anterior ahora representa una variación mayor. Por otro lado,
				  	la variación global es apenas mayor a la del grupo anterior. De todas formas, estos porcentajes están sujetos
				  	a la división arbitraria de los grupos hecha previamente (pues el máximo delta es $500$USD).
				\subsubsection{Grupo 4 - $[1200:1500)$U$\$$D}
					Este cuarto grupo es el segundo grupo principal, concentra el $26\%$ de todos los barrios y es el de menor
					rango de valores, con $300$USD, salvo por el grupo 5, con $250$USD. Esto nos dice que la pendiente en este
					caso será la menor de todas, como se puede ver en el gráfico general. Además, esperamos ver una zona central
					con pendiente casi nula, en donde se encontrará un grupo muy grande de barrios con una variación en el precio
					muy pequeña, a diferencia del grupo 3.
					\begin{center}
   		    				\includegraphics[width=6in, height=4.13in]{images/m2Group4Area}
				  	\end{center}
				  	\tab El subgrupo indicado con verde consiste de dieciocho barrios que se diferencian solamente, entre el primero
				  	y el último, por $53$USD. Como supusimos, en este grupo se encuentra el conjunto de barrios con variación casi
				  	nula más grande de todos. Adicionalmente, si se observa el gráfico con atención, en la mayor parte se encuentran
				  	subgrupos pequeños con valores casi constantes hasta que hay un pequeño descenso de precio, nuevamente un valor
				  	casi constante, etc. dividiendo el grupo en cinco o seis diferentes subgrupos pequeños (o no tanto, como el
				  	verde) por lo que no tiene sentido destacarlos como en el grupo 3, pues solo quedarían aquellos pequeños sectores
				  	donde la pendiente si toma valor considerable. \\
				  	\tab En cuanto al porcentaje de variación, sabiendo que la diferencia entre el máximo local y el mínimo es de
				  	$286$USD, tenemos un porcentaje de variación local del $19\%$ y uno global del $5\%$. Por lo tanto, y como era
				  	de esperarse, este grupo tiene los porcentajes mínimos en ambos casos; en el local porque tiene la mayor cantidad
				  	de segmentos casi constantes y en el global porque, además de la razón recién mencionada, por tener valores
				  	cada vez más bajos. Cabe destacar, además, que el $19\%$ local está concentrado en ciertos 'saltos' de un
				  	barrio a otro; pues, como mencionamos antes, la mayor parte del grupo esta compuesta por subgrupos de precios
				  	similares.
				\subsubsection{Grupo 5 - $[950:1200)$U$\$$D}
					Este quinto y penúltimo grupo comienza una pendiente que incrementa progresivamente hacia el último y sexto
					grupo. Aquí se concentra el $13\%$ de los barrios, la mitad que en el grupo anterior, y los precios de los
					barrios ya se acercan a los minimos que se pueden encontrar.\\
					\tab A partir de la información que brinda el gráfico general, esperamos encontrar una pequeña zona de pendiente
					casi nula hacia el final del grupo y una variación local relativamente chica.
					\begin{center}
   		    				\includegraphics[width=6in, height=4.13in]{images/m2Group5Area}
				  	\end{center}
				  	\tab Indicado con color Cyan en el gráfico podemos ver ese pequeño grupo del que hablábamos donde la pendiente
				  	es casi nula, mientras que en el resto del grupo hay una pendiente casi constante. El tamaño de este subgrupo
				  	es de cinco barrios y la diferencia de precio entre el primero y el último es de $11$USD. \\
				  	\tab Si calculamos el porcentaje de variación, conociendo que la diferencia entre el máximo local y el mínimo
				  	es de $235$USD, obtenemos un porcentaje de variación local del $20\%$ y uno global del $4\%$. En este caso, el
				  	porcentaje local esta distribuido a lo largo de todo el grupo, salvo por el subgrupo final que tiene un valor
				  	casi constante.
				\subsubsection{Grupo 6 - $[450:950)$U$\$$D}
					En este último grupo, que contiene sólo al $9\%$ de los barrios (al igual que el grupo 1), estaremos analizando
					a aquellos barrios con menor precio por $m^2$. A partir del gráfico general sabemos que la pendiente será muy
					marcada y que la diferencia entre un barrio y el siguiente será importante, salvo por una pequeña parte donde
					los precios se mantendrán casi constantes.
					\begin{center}
   		    				\includegraphics[width=6in, height=4.13in]{images/m2Group6Area}
				  	\end{center}
					\tab En amarillo se puede observar la única sección casi constante del grupo, que consta sólo de cuatro barrios
					y que contienen un rango cuya diferencia entre el mínimo y el máximo es de $14$USD. Mientras tanto, el resto
					del grupo tiene una pendiente considerablemente grande y la diferencia entre el valor máximo y el mínimo es casi
					del doble. \\
					\tab Para remarcar esto último, calcularemos el porcentaje de variación sabiendo que dicha diferencia es de
					$410$USD. El porcentaje de variación local es del $45\%$ y el porcentaje de variación global es del $7\%$. Aquí
					la variación local está repartida en todo el grupo pero con mayor participación de la segunda mitad, donde la
					pendiente es mayor, y con menor participación del sector casi constante.\\
		\section{Distribución geográfica}
			En esta sección se mostrará, con la ayuda de \emph{HeatMaps} cómo están distribuídos los precios por $m^2$ en CABA
			y GBA. \\
			\tab Para empezar, mostraremos un mapa general de CABA y GBA donde la unidad de medida es el precio por $m^2$. Se
			debe tener en cuenta que el software utilizado determina el color de un pixel a partir de la suma de todas las
			propiedades que se corresponden con ese pixel, y por dicha razón, podríamos encontrar sectores donde, si bien el
			precio no es tan elevado, hay muchas propiedades y por ende se las indica con rojo. De todas formas, este gráfico
			nos sirve para tener un plano general.
			\begin{center}
				\includegraphics[width=6in, height=2.46in]{images/m2GeneralHeatMap}
		  	\end{center}
			\subsection{Grupos característicos y su ubicación}			
				Ahora, lo que haremos será repetir la dinámica pero dividiendo en los mismo grupos característicos de antes, para
				ver que se puede saber de cada grupo según su ubicación. Cabe destacar que la escala de precios utilizada en los
				mapas de cada grupo es diferente, pues el objetivo aquí es mostrar la ubicación de estos barrios más que el valor
				que ellos tienen.
				\subsubsection{Grupo 1}
					Para el grupo 1, como se dijo previamente, se espera encontrar a las propiedades en CABA y la zona de GBA
					que corresponde al inmediato Norte de la Capital Federal.
					\begin{center}
						\includegraphics[width=6in, height=2.46in]{images/m2Group1HeatMap}
				  	\end{center}
				  	\tab En el gráfico se observa, más específicamente, que las propiedades con valores altos están ubicadas
				  	en el Este de la ciudad, esparciéndose desde el Microcentro hacia el Norte de CABA. Se puede ver, también,
				  	que las propiedades de la Provincia que se encuentran en este grupo son aquellas cercanas al río. Cabe destacar
				  	que Puerto Madero genera una 'discontinuidad' en el mapa, pues si se observa atentamente se ve que no está en
				  	una zona contigua al resto de los barrios. De todos modos, esto era de esperarse porque es un barrio construído
				  	específicamente como un 'lugar caro'. \\
				  	\tab Se debe mencionar, además, que el Barrio El Golf está ubicado en la localidad de Tigre y en este mapa no
				  	se lo puede ver. Se decidió dejarlo afuera para poder dar una mejor vista de la Capital Federal.
				\subsubsection{Grupo 2}
					Este segundo grupo se espera que esté compuesto por los barrios de capital y Zona Norte que no estaban presentes
					en el grupo previo. Esto es, San Isidro, Caballito, Saavedra, etc.
					\begin{center}
						\includegraphics[width=6in, height=2.46in]{images/m2Group2HeatMap}
				  	\end{center}
				  	\tab El gráfico nos muestra que este segundo grupo contiene a los barrios del Sur Este de la Capital Federal
				  	('del centro para abajo', al revés que el grupo 1) y aquellos barrios que 'completan' la Zona Norte. Además, se
				  	encuentran algunos otros barrios dispersos, como por ejemplo Boedo. \\
				  	\tab Cabe destacar aquí también que en este segundo grupo se encuentran más barrios cerrados de la localidad de
				  	Tigre, como el Barrio Santa Bárbara y el Barrio Los Alisos, que nuevamente no se incluyeron en el mapa para
				  	poder dar más detalle a CABA.
				\subsubsection{Grupo 3}
					En este tercer grupo se espera que, además de que se cubra la mayor cantidad de superficie, se 'complete' la
					capital y comiencen a aparecer los barrios del llamado primer cordón del conurbano.
					\begin{center}
						\includegraphics[width=6in, height=2.46in]{images/m2Group3HeatMap}
				  	\end{center}
				  	\tab El mapa presentado nos muestra, efectivamente, que está compuesto de los barrios restantes de la Capital
				  	Federal, algunos barrios del primer cordón, como Villa Martelli, Munro, Morón y otros más lejanos como Quilmes y
				  	Lomas de Zamora. Además, como se puede observar al Norte, se sigue completando la zona de barrios cerrados de la
				  	localidad de Tigre y sus alrededores. \\
				  	\tab También forma parte de este grupo la ciudad de La Plata pero se decidió dejarlo afuera de la visualización
				  	para poder enfocarnos más en CABA y GBA.
				\subsubsection{Grupo 4}
					Para este cuarto grupo, el segundo grupo principal como fue llamado previamente, se espera que se 'complete' el
					primer cordón del conurbano y comiencen a aparecer barrios del segundo y tercer cordón. También se espera que ya
					no aparezcan barrios de CABA.
					\begin{center}
						\includegraphics[width=6in, height=2.46in]{images/m2Group4HeatMap}
				  	\end{center}
				  	\tab En este caso se debió utilizar un nivel de zoom menor por la distancia entre los barrios del grupo. De todas
				  	maneras, se puede apreciar la aparición de más barrios del primer cordón del conurbano y de Pilar y Escobar, que
				  	son, al igual que Tigre, dos centros importantes de barrios cerrados y country clubs.
				\subsubsection{Grupo 5}
					Ya acercándonos a los barrios más baratos esperamos, tanto en este grupo como en el siguiente, encontrar
					barrios alejados de la Capital Federal pertenecientes al segundo y tercer cordón del conurbano.
					\begin{center}
						\includegraphics[width=6in, height=2.46in]{images/m2Group5HeatMap}
				  	\end{center}
				  	\tab Se puede observar que se completan algunas zonas faltantes del primer cordón del conurbano y aparecen zonas
				  	densas del segundo y tercer cordón, como se esperaba. De todas formas, al poseer cada vez menor cantidad de
				  	barrios, es más dificil hacer un mapa claro.
				\subsubsection{Grupo 6}
					Finalmente, en este sexto grupo esperamos encontrar barrios dispersos y alejados de la Capital Federal, con poca
					cantidad de propiedades.
					\begin{center}
						\includegraphics[width=6in, height=2.46in]{images/m2Group6HeatMap}
				  	\end{center}
				  	\tab Se verifica con este gráfico, entonces, que los integrantes de este último grupo pertencen al tercer,
				  	y más alejado, cordón del conurbano bonaerense.
				\subsubsection{Comparación de grupos}
					En este último gráfico mostraremos cómo se distribuyen los grupos en CABA y GBA, unificando todos los gráficos
					previos para un mejor entendimiento. Se utilizaron diferentes colores para diferenciar los grupos y dichos
					colores están indicados en la leyenda del gráfico.
					\begin{center}
						\includegraphics[width=6in, height=2.46in]{images/m2GroupComparison}
				  	\end{center}
				  	\tab NOTA: Grupo 1 AZUL, Grupo 2 VERDE, Grupo 3 ROJO, Grupo 4 AMARILLO, Grupo 5 CYAN, Grupo 6 MAGENTA.
				  	\emph{AGREGARLE LA LEYENDA A LA IMAGEN}
\end{document}